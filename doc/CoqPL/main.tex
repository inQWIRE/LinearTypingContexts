%% For double-blind review submission, w/o CCS and ACM Reference (max submission space)
%\documentclass[sigplan,9pt,review,anonymous]{acmart}\settopmatter{printfolios=true,printccs=false,printacmref=false}
%% For double-blind review submission, w/ CCS and ACM Reference
%\documentclass[sigplan,10pt,review,anonymous]{acmart}\settopmatter{printfolios=true}
% For single-blind review submission, w/o CCS and ACM Reference (max submission space)
\documentclass[sigplan,9pt]{acmart}
\settopmatter{printfolios=true,printccs=false,printacmref=false}
%% For single-blind review submission, w/ CCS and ACM Reference
%\documentclass[sigplan,10pt,review]{acmart}\settopmatter{printfolios=true}
%% For final camera-ready submission, w/ required CCS and ACM Reference
%\documentclass[sigplan,10pt]{acmart}\settopmatter{}


%% Conference information
%% Supplied to authors by publisher for camera-ready submission;
%% use defaults for review submission.
\acmConference[CoqPL'17]{CoqPL 2018 The Fourth International Workshop on Coq for Programming Languages}{January 13, 2018}{New York, NY, USA}
\acmYear{2018}
\acmISBN{} % \acmISBN{978-x-xxxx-xxxx-x/YY/MM}
\acmDOI{} % \acmDOI{10.1145/nnnnnnn.nnnnnnn}
\startPage{1}

%% Copyright information
%% Supplied to authors (based on authors' rights management selection;
%% see authors.acm.org) by publisher for camera-ready submission;
%% use 'none' for review submission.
\setcopyright{none}
%\setcopyright{acmcopyright}
%\setcopyright{acmlicensed}
%\setcopyright{rightsretained}
%\copyrightyear{2017}           %% If different from \acmYear

%% Bibliography style
\bibliographystyle{ACM-Reference-Format}
%% Citation style
%\citestyle{acmauthoryear}  %% For author/year citations
%\citestyle{acmnumeric}     %% For numeric citations
%\setcitestyle{nosort}      %% With 'acmnumeric', to disable automatic
                            %% sorting of references within a single citation;
                            %% e.g., \cite{Smith99,Carpenter05,Baker12}
                            %% rendered as [14,5,2] rather than [2,5,14].
%\setcitesyle{nocompress}   %% With 'acmnumeric', to disable automatic
                            %% compression of sequential references within a
                            %% single citation;
                            %% e.g., \cite{Baker12,Baker14,Baker16}
                            %% rendered as [2,3,4] rather than [2-4].


%%%%%%%%%%%%%%%%%%%%%%%%%%%%%%%%%%%%%%%%%%%%%%%%%%%%%%%%%%%%%%%%%%%%%%
%% Note: Authors migrating a paper from traditional SIGPLAN
%% proceedings format to PACMPL format must update the
%% '\documentclass' and topmatter commands above; see
%% 'acmart-pacmpl-template.tex'.
%%%%%%%%%%%%%%%%%%%%%%%%%%%%%%%%%%%%%%%%%%%%%%%%%%%%%%%%%%%%%%%%%%%%%%


%% Some recommended packages.
\usepackage{booktabs}   %% For formal tables:
                        %% http://ctan.org/pkg/booktabs
\usepackage{subcaption} %% For complex figures with subfigures/subcaptions
                        %% http://ctan.org/pkg/subcaption
%\usepackage{bigfoot} % helps listings go in footnotes
\usepackage{xcolor}
\hypersetup{
    colorlinks,
    linkcolor={red!50!black},
    citecolor={blue!50!black},
    urlcolor={blue!80!black}
}
\usepackage[capitalize,noabbrev]{cleveref} % must be loaded after hyperref

\usepackage[utf8]{inputenc}
%\usepackage[T1]{fontenc}
%\newcommand{\upsidedownbang}{!`}
\newcommand{\upsidedownbang}{\rotatebox[origin=c]{180}{!}}
%\usepackage{anyfontsize}
\usepackage{xspace}

%\usepackage{mathastext}
%\MTfamily{\ttdefault}\Mathastext % reload, but use \ttdefault


\usepackage{amsmath,amsfonts,amsthm}

% Figures
\usepackage{wrapfig}

% Lists
\usepackage[inline,shortlabels]{enumitem}

% Proof trees
\usepackage{mathpartir}

% Symbolx
\usepackage{cmll} % Linear logic symbols
\usepackage{ stmaryrd } % more math symbols
\usepackage{upquote} % straight quote

% Bibliography
%\usepackage[numbers]{natbib}
%\usepackage{url}
%\usepackage[strings]{underscore}


% Quantum
\usepackage[qm,braket]{qcircuit}
%\usepackage{braket}
\newcommand{\qwire}{\ensuremath{\mathcal{Q}\textsc{wire}}\xspace}
\newcommand{\liquid}{LIQ\emph{Ui}$\ket{}$}

%   Tikz
\usepackage{tikz} %commutativity diagrams
\usetikzlibrary{%
  arrows,%
  shapes.misc,% wg. rounded rectangle
  shapes.arrows,%
  shapes.callouts,
  chains,%
  matrix,%
  positioning,% wg. " of "
  scopes,%
  decorations.pathmorphing,% /pgf/decoration/random steps | erste Graphik
  decorations.text,
  shadows%
}
\tikzset{ machine/.style={
    % The shape:
    rectangle,
    % The size:
    minimum width=20mm,
    minimum height=10mm,
    text width=16mm,
    % The alignment
    align=center,
    % The border:
    very thick,
    draw=black,
    % The colors:
    color=black,
    fill=white,
    % Font
    font=\ttfamily,
  }
}

\newcommand{\function}[2]{#1~#2}
\newcommand*\rfrac[2]{{}^{#1}\!/_{#2}}


% Latin  Abbr
\newcommand{\etal}{\emph{et al.}\xspace}
\newcommand{\eg}{\emph{e.g.,}\xspace}
\newcommand{\ie}{\emph{i.e.,}\xspace}
\newcommand{\etc}{\emph{etc.}\xspace}


% Functions
\newcommand{\lbind}{\ensuremath{\mathbin{=\!\!>\!\!>\!\!=}}}
\newcommand{\fun}{$\lambda$}
\renewcommand{\put}[1]{\function{\code{put}}{#1}}
\newcommand{\letbang}[3]{\code{let!}~#1=#2~\code{in}~#3}
\newcommand{\suspend}[1]{\function{\code{suspend}}{#1}}
\newcommand{\force}[1]{\function{\code{force}}{#1}}
\newcommand{\open}[1]{\function{\code{open}}{#1}}
\newcommand{\close}[1]{\function{\code{close}}{#1}}

\newcommand{\denote}[1]{\ensuremath{\llbracket#1\rrbracket}}



% Table formatting
\newcommand{\ltwo}[1]{\multicolumn{2}{l}{#1}}
\newcommand{\ctwo}[1]{\multicolumn{2}{c}{#1}}
\usepackage{booktabs}

% Types
\newcommand{\One}{\code{One}}
\newcommand{\Bit}{\code{Bit}}
\newcommand{\Qubit}{\code{Qubit}}
% Qwire Syntax
\newcommand{\Gate}[2]{\code{Gate}(#1,#2)}
\newcommand{\Unitary}[2]{\code{Unitary}(#1,#2)}
\renewcommand{\Box}[2]{\code{Box}(#1,#2)}
\newcommand{\mkbox}[2]{\code{box}(#1 \Rightarrow #2)}
\newcommand{\qcontrol}[1]{\code{control}~#1}  % \control conflicts with qcircuit package
\newcommand{\bitcontrol}[1]{\code{bit-control}~#1}

% Code
\newcommand{\None}{\code{None}}
\newcommand{\Some}[1]{\code{Some}~#1}


% Listings
\usepackage{listings,lstcoq}
%\usepackage{MnSymbol}
\definecolor{ltblue}{rgb}{0,0.4,0.4}
\definecolor{dkblue}{rgb}{0,0.1,0.6}
\definecolor{dkgreen}{rgb}{0,0.35,0}
\definecolor{dkviolet}{rgb}{0.3,0,0.5}
\definecolor{dkred}{rgb}{0.5,0,0}
\lstset{language=Coq}

\newcommand{\code}[1]{{\small\texttt{#1}}}
%\newcommand{\code}[1]{\lstinline{#1}}


\newenvironment{nscenter}
 {\parskip=3pt\par\nopagebreak\centering}
 {\par\noindent\ignorespacesafterend}

%% Unicode
\usepackage{newunicodechar}
\let\Alpha=A
\let\Beta=B
\let\Epsilon=E
\let\Zeta=Z
\let\Eta=H
\let\Iota=I
\let\Kappa=K
\let\Mu=M
\let\Nu=N
\let\Omicron=O
\let\omicron=o
\let\Rho=P
\let\Tau=T
\let\Chi=X

\newunicodechar{Α}{\ensuremath{\Alpha}}
\newunicodechar{α}{\ensuremath{\alpha}}
\newunicodechar{Β}{\ensuremath{\Beta}}
\newunicodechar{β}{\ensuremath{\beta}}
\newunicodechar{Γ}{\ensuremath{\Gamma}}
\newunicodechar{γ}{\ensuremath{\gamma}}
\newunicodechar{Δ}{\ensuremath{\Delta}}
\newunicodechar{δ}{\ensuremath{\delta}}
\newunicodechar{Ε}{\ensuremath{\Epsilon}}
\newunicodechar{ε}{\ensuremath{\epsilon}}
\newunicodechar{ϵ}{\ensuremath{\varepsilon}}
\newunicodechar{Ζ}{\ensuremath{\Zeta}}
\newunicodechar{ζ}{\ensuremath{\zeta}}
\newunicodechar{Η}{\ensuremath{\Eta}}
\newunicodechar{η}{\ensuremath{\eta}}
\newunicodechar{Θ}{\ensuremath{\Theta}}
\newunicodechar{θ}{\ensuremath{\theta}}
\newunicodechar{ϑ}{\ensuremath{\vartheta}}
\newunicodechar{Ι}{\ensuremath{\Iota}}
\newunicodechar{ι}{\ensuremath{\iota}}
\newunicodechar{Κ}{\ensuremath{\Kappa}}
\newunicodechar{κ}{\ensuremath{\kappa}}
\newunicodechar{Λ}{\ensuremath{\Lambda}}
\newunicodechar{λ}{\ensuremath{\lambda}}
\newunicodechar{Μ}{\ensuremath{\Mu}}
\newunicodechar{μ}{\ensuremath{\mu}}
\newunicodechar{Ν}{\ensuremath{\Nu}}
\newunicodechar{ν}{\ensuremath{\nu}}
\newunicodechar{Ξ}{\ensuremath{\Xi}}
\newunicodechar{ξ}{\ensuremath{\xi}}
\newunicodechar{Ο}{\ensuremath{\Omicron}}
\newunicodechar{ο}{\ensuremath{\omicron}}
\newunicodechar{Π}{\ensuremath{\Pi}}
\newunicodechar{π}{\ensuremath{\pi}}
\newunicodechar{ϖ}{\ensuremath{\varpi}}
\newunicodechar{Ρ}{\ensuremath{\Rho}}
\newunicodechar{ρ}{\ensuremath{\rho}}
\newunicodechar{ϱ}{\ensuremath{\varrho}}
\newunicodechar{Σ}{\ensuremath{\Sigma}}
\newunicodechar{σ}{\ensuremath{\sigma}}
\newunicodechar{ς}{\ensuremath{\varsigma}}
\newunicodechar{Τ}{\ensuremath{\Tau}}
\newunicodechar{τ}{\ensuremath{\tau}}
\newunicodechar{Υ}{\ensuremath{\Upsilon}}
\newunicodechar{υ}{\ensuremath{\upsilon}}
\newunicodechar{Φ}{\ensuremath{\Phi}}
\newunicodechar{φ}{\ensuremath{\phi}}
\newunicodechar{ϕ}{\ensuremath{\varphi}}
\newunicodechar{Χ}{\ensuremath{\Chi}}
\newunicodechar{χ}{\ensuremath{\chi}}
\newunicodechar{Ψ}{\ensuremath{\Psi}}
\newunicodechar{ψ}{\ensuremath{\psi}}
\newunicodechar{Ω}{\ensuremath{\Omega}}
\newunicodechar{ω}{\ensuremath{\omega}}

\newunicodechar{ℂ}{\ensuremath{\mathbb{C}}}
\newunicodechar{ℕ}{\ensuremath{\mathbb{N}}}
\newunicodechar{∅}{\ensuremath{\emptyset}}

\newunicodechar{•}{\ensuremath{\bullet}}
\newunicodechar{≈}{\ensuremath{\approx}}
\newunicodechar{≅}{\ensuremath{\cong}}
\newunicodechar{≡}{\ensuremath{\equiv}}
\newunicodechar{≤}{\ensuremath{\le}}
\newunicodechar{≥}{\ensuremath{\ge}}
\newunicodechar{≠}{\ensuremath{\neq}}
\newunicodechar{∀}{\ensuremath{\forall}}
\newunicodechar{∃}{\ensuremath{\exists}}
\newunicodechar{±}{\ensuremath{\pm}}
\newunicodechar{∓}{\ensuremath{\pm}}
\newunicodechar{·}{\ensuremath{\cdot}}
\newunicodechar{⋯}{\ensuremath{\cdots}}
\newunicodechar{…}{\ensuremath{\ldots}}
\newunicodechar{∷}{~\mathrel{:\!\!\!:}~}
\newunicodechar{×}{\ensuremath{\times}}
\newunicodechar{∞}{\ensuremath{\infty}}
\newunicodechar{→}{\ensuremath{\to}}
\newunicodechar{←}{\ensuremath{\leftarrow}}
\newunicodechar{↔}{\ensuremath{\leftrightarrow}}
\newunicodechar{⇒}{\ensuremath{\Rightarrow}}
\newunicodechar{↦}{\ensuremath{\mapsto}}
\newunicodechar{↝}{\ensuremath{\leadsto}}
\newunicodechar{∨}{\ensuremath{\vee}}
\newunicodechar{∧}{\ensuremath{\wedge}}
\newunicodechar{⊢}{\ensuremath{\vdash}}
\newunicodechar{⊣}{\ensuremath{\dashv}}
\newunicodechar{∣}{\ensuremath{\mid}}
\newunicodechar{∈}{\ensuremath{\in}}
\newunicodechar{⊆}{\ensuremath{\subseteq}}
\newunicodechar{⋓}{\ensuremath{\Cup}}
\newunicodechar{∩}{\ensuremath{\cap}}
\newunicodechar{∪}{\ensuremath{\cup}}
\newunicodechar{∉}{\ensuremath{\not\in}}

\newunicodechar{⊸}{\ensuremath{\multimap}}
\newunicodechar{⊗}{\ensuremath{\otimes}}
\newunicodechar{⊕}{\ensuremath{\oplus}}
\newunicodechar{〈}{\ensuremath{\langle}}
\newunicodechar{⟨}{\ensuremath{\langle}}
\newunicodechar{⟩}{\ensuremath{\rangle}}
\newunicodechar{〉}{\ensuremath{\rangle}}
\newunicodechar{¡}{\ensuremath{\upsidedownbang}}
\newunicodechar{∘}{\ensuremath{\circ}}
\newunicodechar{∙}{\ensuremath{\bullet}}
\newunicodechar{†}{\ensuremath{\dagger}}
\newunicodechar{⊤}{\ensuremath{\top}}
\newunicodechar{⊥}{\ensuremath{\bot}}

\newunicodechar{〚}{\ensuremath{\llbracket}}
\newunicodechar{〛}{\ensuremath{\rrbracket}}


%% COMMENTS 
\usepackage{etoolbox} % replaces ifthen package
\newtoggle{comments}
\toggletrue{comments}
%\togglefalse{comments}

\iftoggle{comments}{
  \newcommand{\jennifer}[1]{\textbf{\textcolor{red}{[ #1 --- Jennifer]}}}
  \newcommand{\steve}[1]{\textbf{\textcolor{green}{[ #1 --- Steve]}}}
  \newcommand{\robert}[1]{\textbf{\textcolor{blue}{[ #1 --- Robert]}}}

  \newcommand{\fixme}[1]{\textbf{\textcolor{red}{[ Fixme: #1]}}}
  \newcommand{\note}[1]{\textbf{\textcolor{green}{[ Note: #1 ]}}}
  \newcommand{\todo}[1]{\textbf{\textcolor{green}{[ TODO: #1 ]}}}
}{
  \newcommand{\jennifer}[1]{}
  \newcommand{\steve}[1]{}
  \newcommand{\robert}[1]{}

  \newcommand{\fixme}[1]{}
  \newcommand{\note}[1]{}
  \newcommand{\todo}[1]{}
}
%% END COMMENTS

%%% Local Variables:
%%% mode: latex
%%% TeX-master: "main"
%%% End:


\begin{document}

%% Title information
\title{Linear Typing Judgments in Coq}         %% [Short Title] is optional;



%% Author information
%% Contents and number of authors suppressed with 'anonymous'.
%% Each author should be introduced by \author, followed by
%% \authornote (optional), \orcid (optional), \affiliation, and
%% \email.
%% An author may have multiple affiliations and/or emails; repeat the
%% appropriate command.
%% Many elements are not rendered, but should be provided for metadata
%% extraction tools.

%% Author with single affiliation.

\author{Jennifer Paykin}
\affiliation{
  \institution{University of Pennsylvania}            %% \institution is required
}
\email{jpaykin@seas.upenn.edu}          %% \email is recommended

\author{Robert Rand}
\affiliation{
  \institution{University of Pennsylvania}            %% \institution is required
}
\email{rrand@seas.upenn.edu}          %% \email is recommended

\author{Steve Zdancewic}
\affiliation{
  \institution{University of Pennsylvania}            %% \institution is required
}
\email{stevez@cis.upenn.edu}          %% \email is recommended

%% Abstract
%% Note: \begin{abstract}...\end{abstract} environment must come
%% before \maketitle command
\begin{abstract}
  Stuff
\end{abstract}


%% 2012 ACM Computing Classification System (CSS) concepts
%% Generate at 'http://dl.acm.org/ccs/ccs.cfm'.
%\begin{CCSXML}
%<ccs2012>
%<concept>
%<concept_id>10011007.10011006.10011008</concept_id>
%<concept_desc>Software and its engineering~General programming languages</concept_desc>
%<concept_significance>500</concept_significance>
%</concept>
%<concept>
%<concept_id>10003456.10003457.10003521.10003525</concept_id>
%<concept_desc>Social and professional topics~History of programming languages</concept_desc>
%<concept_significance>300</concept_significance>
%</concept>
%</ccs2012>
%\end{CCSXML}

%\ccsdesc[500]{Software and its engineering~General programming languages}
%\ccsdesc[300]{Social and professional topics~History of programming languages}
%% End of generated code


%% Keywords
%% comma separated list
%\keywords{formal verification, type theory, quantum computing}  %% \keywords are mandatory in final camera-ready submission


%% \maketitle
%% Note: \maketitle command must come after title commands, author
%% commands, abstract environment, Computing Classification System
%% environment and commands, and keywords command.
\maketitle

\section{Introduction}

When defining a simple type system in Coq, there are a large number of design
decisions to make. What representation of variable bindings should I use? What
representation of typing contexts? What about substitution? After all that, the
typing judgment itself is usually not all that hard to write down and it should
be fairly easy to (manually) prove that a particular term is well-typed.
\jennifer{But actually, these are not the questions we are considering here.}

Linear typing judgments add another layer of complexity to the equation.
Consider a simple linear lambda calculus:
\[
    \inferrule*
    {~}
    {x:τ ⊢ x : τ}
\quad
    \inferrule*
    {Γ ⋓ x:σ ⊢ e : τ}
    {Γ ⊢ λx.e : σ ⊸ τ}
\quad
    \inferrule*
    {Γ_1 ⊢ e_1 : σ ⊸ τ \\ Γ_2 ⊢ e_2 : σ}
    {Γ_1 ⋓ Γ_2 ⊢ e_1 e_2 : τ}
\]

There are a few important differences from the simply-typed lambda calculus,
stemming from the fact that linear variables can neither be discarded or
duplicated.\footnote{We could also consider other substructural type systems,
  which consider different subsets of these rules, and although the details
  differ, many of the same principles apply.} First, in the variable rule, the
typing context must be exactly the singleton context $x:τ$; the corresponding
STLC rule, where $Γ ⊢ x : τ$ provided $(x:τ) ∈ Γ$, might discard additional
variables in $Γ$, contradicting linearity. Second, in the abstraction and
application rules, we write $Γ_1 ⋓ Γ_2$ for the \emph{partial, disjoint merge}
of $Γ_1$ and $Γ_2$, which is undefined if $Γ_1 ∩ Γ_2 ≠ ∅$. This ensures that
linear variables cannot be duplicated to be used on both sides of an
application, or shadowed in the abstraction rule.

Unfortunately, type checking this judgment is not so straightforward. To show
that $Γ ⊢ e_1 e_2 : τ$, the typing context $Γ$ must be ``split'' in two disjoint
parts, depending on which variables occur in which parts. The ordinary
approach to linear typing judgments is to thread usage information through the
typing judgment. For example, one might instead consider a typing judgment
$Γ_{\text{in}} ⊢ e : τ ⟩ Γ_{\text{out}}$ where $e$ uses exactly the variables in
$Γ_{\text{in}} \textbackslash Γ_{\text{out}}$, meaning that $Γ_{\text{in}}$ is
thought of as ``input'' to the judgment, and $Γ_{\text{out}}$ is thought of as
``output''. This approach works well for affine type systems (where linear
variables may be discarded but not duplicated) but to enforce true linearity,
one must consider a top-level judgment $Γ ⊢ e : τ ⟩ ∅$. In addition, this kind
of typing structure is not particularly elegant for additive types \todo{cite},
and requires serializing the typing judgment. For example, to type an
application $e_1 e_2$, should it really matter if we type $e_1$ or $e_2$ first?

On the whole, we would prefer a judgment in Coq corresponding to the natural
presentation of linear types: $Γ ⊢ e : τ$, asserting that $e$ uses
\emph{exactly} the variables in $Γ$. Our goal in this work is to define a linear
typing judgment \coqe{Has_LType : Ctx -> Ty -> Prop} in Coq and be able to
easily (automatically) discharge constraints stemming from the linearity of
typing contexts. In addition, the technique should not rely on a particular
representation of variable binding or typing contexts.

\section{Partiality}

Partial functions in Coq may take a number of forms. A partial function might be
represented as an inductive relation, which has the disadvantage that the result
cannot be computed on concrete inputs. On the other hand, a partial function
might be represented as a function to the option type, which has the
disadvantage of obscuring compositionality, since composition must be surrounded
by the monadic bind. A third option is to lift the partiality to all data
values, so that disjoint merge is total, but may produce an ``undefined''
typing context as a result. This third option is the approach we take in this paper.
\jennifer{I don't think this paragraph is super clear}

Thus, we can define the linear typing judgment in Coq as follows:
\begin{coq}
Inductive Has_LType : Ctx -> Ty -> Prop :=
| L_Var : ∀ Γ x τ, Γ = singleton x τ ->
                   Has_LType Γ (Var x) τ
| L_Abs : ∀ Γ x e σ, Γ ⋓ singleton x σ ≠ undefined ->
                     Has_LType (Γ ⋓ singleton x σ) e τ ->
                     Has_LType Γ (Abs x e) (σ ⊸ τ)
| L_App : ∀ Γ1 Γ2 Γ e1 e2 σ τ, Γ1 ⋓ Γ2 = Γ -> Γ ≠ undefined ->
                               Has_LType Γ1 e1 (σ ⊸ τ) ->
                               Has_LType Γ2 e2 σ ->
                               Has_LType Γ (App e1 e2) τ.
\end{coq}

\todo{As a running example, try to type something.}



%% Acknowledgments
\begin{acks}                            %% acks environment is optional
                                        %% contents suppressed with 'anonymous'
  %% Commands \grantsponsor{<sponsorID>}{<name>}{<url>} and
  %% \grantnum[<url>]{<sponsorID>}{<number>} should be used to
  %% acknowledge financial support and will be used by metadata
  %% extraction tools.
  This material is based upon work supported by the
  \grantsponsor{GS100000001}{National Science
    Foundation}{http://dx.doi.org/10.13039/100000001} under Grant
  No.~\grantnum{GS100000001}{nnnnnnn} and Grant
  No.~\grantnum{GS100000001}{mmmmmmm}.  Any opinions, findings, and
  conclusions or recommendations expressed in this material are those
  of the author and do not necessarily reflect the views of the
  National Science Foundation.
\end{acks}


%% Bibliography
\bibliography{bibliography}


\end{document}

%%% Local Variables:
%%% mode: latex
%%% TeX-master: t
%%% End:
